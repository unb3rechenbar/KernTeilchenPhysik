\documentclass{subfiles}

\begin{document}

\subaufgabe{}
\begin{itemize}
        \item Ansatz: prozentuelles Zerfallsgesetz (normiert auf die die Gesamtmenge der Uranprobe \textbf{heute}):
        \begin{align*}
                H_{235}(t) &= H_{235,0}\cdot\exp(-\frac{\ln(2)}{\tau_{235}}\cdot t),\\
                H_{238}(t) &= H_{238,0}\cdot\exp(-\frac{\ln(2)}{\tau_{238}}\cdot t).
        \end{align*}
        $H$ bezeichnet die jeweilige Häufigkeit, also $H_{235}(T) = \SI{99.28}{\percent}$, $H_{238}(T) = \SI{0.72}{\percent}$, wobei $T=\SI{600e6}{a}$ der Messzeitpunkt ist. Halbwertszeiten: $\tau_{235}=\SI{7.038e8}{a}$, $\tau_{238}=\SI{4.468e9}{a}$, Umgestellt nach ursprünglichen Häufigkeiten:
        \begin{align*}
                H_{235,0} &= H_{235}(T)\cdot\exp(\frac{\ln(2)}{\tau_{235}}\cdot t),\\
                H_{238,0} &= H_{238}(T)\cdot\exp(\frac{\ln(2)}{\tau_{238}}\cdot t).
        \end{align*}
        Einsetzen liefert $H_{235,0} = \SI{1.30}{\percent}$, $H_{238,0} = \SI{108.96}{\percent}$.
        \item Normierung auf damalige Häufigkeit $H'$ gelingt durch Division mit $S=H_{235,0} + H_{238,0}$: \underline{$H'_{235,0}=\SI{1.20}{\percent}$ und $H'_{238,0}=\SI{98.8}{\percent}$}.
\end{itemize}

\end{document}